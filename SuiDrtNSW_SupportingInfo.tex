% NOTES
% ONLY THE DROUGHT INDEX CODES WILL RUN UNLESS YOU ARE AN AUTHORISED USER OF AUSTRALIAN SUICIDE DATA
% THEREFORE THE ENTIRE SWEAVE FILE CANNOT BE COMPILED UNTIL ACCESS IS ARRANGED
% USERS ARE RECOMMENDED TO REPLACE eval=FALSE with TRUE
% AND THEN USE > Stangle('SuiDrtNSW_SupportingInfo.Rnw') TO CREATE AN R FILE FIRST
% THEN RUN EACH R CHUNK INDIVIDUALLY (AND NOTE INSTRUCTIONS IN THE COMMENTS)
% SORRY FOR THE INCONVENIENCE AND THANKYOU FOR YOUR PATIENCE
% IVAN HANIGAN, 2012-05-28
% PS THIS IS THE ABRIDGED VERSION, IT DIFFERS TO THE UNABRIDGED SI APPENDIX ONLY BY COMMENTED SECTIONS LABELLED 'UNABRIDGED'

%%%%%%%%%%%%%%%%%%%% PREAMBLE BEGINS %%%%%%%%%%%%%%%%%%%%
\documentclass[a4paper]{article}                % Default font size = 10pt
\usepackage[top=1.5in,bottom=1in,left=1.25in,right=1.25in]{geometry}
%% For font size >10pt, set margins to 1in, or use 'fullpage' package
%% instead of 'geometry' package
%\usepackage{fullpage}
\usepackage[T1]{fontenc}
\usepackage{ae,aecompl}
\usepackage{amsmath,amsfonts,amssymb}  % Amer Math Society stuff
\usepackage{graphicx}                  % Enables \includegraphics
\usepackage{setspace}                  % Enables doublespacing
\usepackage{tabularx}                  % Better tables
\usepackage{verbatim}                  % Enables comment environment
\usepackage{cite}                      % Citations appear [1, 2-7, 9]
\usepackage{url}                       % Proper display of URLs
\usepackage{hyperref}                 % Enables hyperlinks
%\usepackage{breakurl}                 % Proper linebreaking of URLs

%%%%%%%%%% Miscellaneous commands %%%%%%%%%%
% \pagestyle{headings}

%%%%%%%%%% User-defined commands %%%%%%%%%%
\usepackage{Sweave}
%%%%%%%%%%%%%%%%%%%% PREAMBLE ENDS %%%%%%%%%%%%%%%%%%%%

\begin{document}
\input{SuiDrtNSW_SupportingInfo-concordance}
%%++++++++ Title page begins ++++++++%%
\title{Online Supporting Information for the article:\\ ``Suicide and Drought in NSW, Australia, 1970-2007''.}
%\\ Unabridged}
\author{Ivan C. Hanigan$^1$$^,$$^2$ \and Colin D. Butler$^1$ \and
  Philip N. Kokic$^2$ \and
  Michael F. Hutchinson$^3$}
\date{}
\maketitle


\noindent [$^1$]National Centre for Epidemiology and Population Health, Australian National University


\noindent [$^2$]Commonwealth Scientific and Industrial Research Organisation

\noindent [$^3$]Fenner School of Environment and Society, Australian National University.



\setcounter{page}{1}
\pagenumbering{roman}
\tableofcontents
\pagenumbering{arabic}
\setcounter{page}{1}
%%++++++++ Title page ends ++++++++%%

\doublespacing   %% Switching to DOUBLE SPACING




\section{Introduction}

%%\section{Introduction}
        This document accompanies the R code at this website \url{https://github.com/ivanhanigan/SuicideAndDroughtInNSW} to calculate the Hutchinson Drought Index and fit the regression models for the paper `Suicide and Drought in New South Wales (NSW), Australia, 1970-2007'.  The calculation of the Drought Index is demonstrated using free data from the Australian Bureau of Meteorology. The suicide mortality data are not publicly available due to confidentiality restrictions. The R code we ran the regressions with is included but the original data are only available for authorised users approved by the Australian Bureau of Statistics and the NSW Registrar of Births Deaths and Marriages.


%% \subsection{Copyright}




        %################################################################################
        %## Copyright 2011, Ivan C Hanigan <ivan.hanigan@gmail.com> and Michael F Hutchinson
        %## This program is free software; you can redistribute it and/or modify
        %## it under the terms of the GNU General Public License as published by
        %## the Free Software Foundation; either version 2 of the License, or
        %## (at your option) any later version.
        %##
        %## This program is distributed in the hope that it will be useful,
        %## but WITHOUT ANY WARRANTY; without even the implied warranty of
        %## MERCHANTABILITY or FITNESS FOR A PARTICULAR PURPOSE.  See the
        %## GNU General Public License for more details.
        %## Free Software
        %## Foundation, Inc., 51 Franklin Street, Fifth Floor, Boston, MA
        %## 02110-1301, USA
        %################################################################################





\section{Drought Index}

%%\section{Drought Index}
        The R code includes a demonstration of the Hutchinson Drought Index \cite{Smith1992}.  This climatic drought index is shown graphically for a location in the `Central West' SD of NSW in Figure \ref{fig:CentralWestDrought8283.png}.



%% \subsection{Drought tools}


%% \subsection{dlMonthly}



%% \subsection{droughtIndex}


%% \subsection{create download directories}


%% \subsection{Download spatial data}

%% \subsection{Download Weather Data}
        Instructions for using R to download and analyse the spatial data from the Australian Bureau of Statistics (\url{http://www.abs.gov.au}) and the weather data from the Australian Bureau of Meteorology (\url{ http://www.bom.gov.au}) websites are included.


%% \subsection{subset the SDs to NSW}


%% \subsection{subset the SDs to Vic}


%% \subsection{Download the Rainfall Station location data}



%% \subsection{revert to project root dir}


%% \subsection{Plot the NSW SD and stations}


%% \subsection{go for a SD wide average rainfall using these stations}


\subsection{Calculate the Drought Index}

%%\subsection{Calculate the Drought Index}
        The Drought index is shown in Figure \ref{fig:CentralWestDrought8283.png} for the SD of `Central West NSW' during a period which includes a strong drought (1979-83). The raw monthly rainfall totals are integrated to rolling 6-monthly totals (both shown in first panel) which are then ranked into percentiles by month and this is rescaled to range between -4 and +4 in keeping with the range of the Palmer Index \cite{Palmer1965} (second panel).  Mild drought is below -1 in the Palmer index and so consecutive months below this threshold are counted. In the original method 5 or more consecutive months was defined as the beginning of a drought, which continued until the rescaled percentiles exceed -1 again  (third panel).  The enhanced method imposes a more conservative threshold of zero (the median) to break a drought (fourth panel).

        There was also an alternative method devised by Hutchinson where the rescaled percentile values are integrated using conditional cumulative sums.  That method is included in the R code however we decided not to use it in this study because the counting method is simpler and gives similar results.

        \begin{figure}[!h]
        \centering
        \includegraphics[width=1\textwidth]{CentralWestDrought8283.png}
        \caption{The Drought index in Central West NSW with the enhanced method shown in the fourth panel.}
        \label{fig:CentralWestDrought8283.png}
        \end{figure}
        %\clearpage



%% \subsection{replicate Fig3.5 from Hutchinson}


%% \subsection{plot the Victorian SD and stations}


%% \subsection{SD wide average}


%% \subsection{Seymour drought index}


%% \subsection{Integration by Conditional Summation}

 %% unabridged
        % \subsection{The Summation Method}
        % When the index is calculated using the sum of each consecutive month's rainfall deficiency score the resulting measure addresses  the question of how intense the drought is, rather than just the duration which is provided by the counting method.  This version  of the index is shown in Figure \ref{fig:SeymourDrought9499enhanced.png}.


        % \begin{figure}[!h]
        % \centering
        % \includegraphics[width=\textwidth]{SeymourDrought9499enhanced.png}
        % \caption{SeymourDrought9499enhanced.png}
        % \label{fig:SeymourDrought9499enhanced.png}
        % \end{figure}
        % %\clearpage



\section{Suicide and Drought Modeling}

%%\section{Suicide and Drought Modeling}


%% \subsection{Extract preprocessed data from database}


%% \subsection{Load data to R server}


\subsection{Descriptive Statistics of Drought and Suicide}
Descriptive statistics for the Drought Index are shown in Table~1.  Summary statistics for Suicide rates are shown in Table~2.


%% \subsection{Descriptive statistics of Drought}

% TASK PNAS
        % latex table generated in R 2.12.0 by xtable 1.5-6 package
        % Mon Aug 08 09:39:16 2011
        \begin{table}[!ht]
        \begin{center}
        \caption{Descriptive statistics for the drought index}
        \label{tab:tab1}
        \begin{tabular}{lrrr}
        \hline
        SD group & N droughts & Avg Duration & Max Duration \\
        \hline
        1 Central West & 9 & 8 & 12 \\
        2 Hunter & 11 & 7 & 15 \\
        3 Illawarra & 7 & 9 & 16 \\
        4 Mid-North Coast & 8 & 8 & 15 \\
        5 Murray & 7 & 8 & 11 \\
        6 Murrumbidgee & 10 & 7 & 11 \\
        7 North and Far Western & 8 & 7 & 12 \\
        8 Northern & 5 & 8 & 11 \\
        9 Richmond-Tweed & 13 & 8 & 17 \\
        10 South Eastern & 8 & 8 & 11 \\
        11 Sydney & 9 & 9 & 20 \\
        \hline
        \end{tabular}
        \end{center}
        \end{table}


%% \subsection{Descriptive statistics of Suicide}

% TASK PNAS
        % latex table generated in R 2.12.0 by xtable 1.5-6 package
        % Mon Aug 08 09:39:16 2011
        \begin{table}[!ht]
        \begin{center}
        \caption{Descriptive statistics for suicide (PYL = Person Years Lived)}
        \label{tab:tab2}
        \begin{tabular}{lrrr}
        \hline
        SD group & Avg Death/Month & Avg Pop & Rate/100000 PYL \\
        \hline
        1 Central West & 2 & 138202 & 13 \\
        2 Hunter & 5 & 430403 & 13 \\
        3 Illawarra & 3 & 280037 & 13 \\
        4 Mid-North Coast & 2 & 183521 & 12 \\
        5 Murray & 1 & 86221 & 14 \\
        6 Murrumbidgee & 1 & 118778 & 13 \\
        7 North and Far Western & 2 & 114460 & 16 \\
        8 Northern & 2 & 146465 & 14 \\
        9 Richmond-Tweed & 2 & 139356 & 14 \\
        10 South Eastern & 2 & 135091 & 14 \\
        11 Sydney & 34 & 3040952 & 13 \\
        \hline
        \end{tabular}
        \end{center}
        \end{table}

\clearpage

\subsection{Correlation between Temperature and Drought}

 We found that monthly maximum temperature variables are not strongly correlated with the drought index in our dataset.  Correlation coefficients for the variables are shown in Table 3.

        %Table \ref{tab:Correlations}


